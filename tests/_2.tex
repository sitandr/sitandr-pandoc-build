\section{Операции над обобщенными функциями}

\(D'  (\mathbb{R} )\) - линейное пространство.

\begin{enumerate}
\def\labelenumi{\arabic{enumi}.}
\item
  \textbf{Умножение на гладкую}

  Аналогия с регулярными: \[
  f\in L_{1, loc}, \alpha  \in  C^\infty  \\
  (\alpha f, \varphi ) = \int\limits_R \alpha \cdot f\varphi \cdot dx=(f, \alpha \varphi )
  \]

  \begin{definition}

  Пусть \(f  \in  D'  (\mathbb{R} )\),
  \(\alpha   \in  C^{\infty }  (\mathbb{R} )\)
  \((\alpha  f, \varphi )  =  (f, \alpha  \varphi )\).

  \end{definition}

  Корректность:
  \(\varphi   \in  D  (\mathbb{R} )  \;  \Rightarrow   \; \alpha  \varphi   \in  D  (\mathbb{R} )\).
  Линейность --- ясно. Непрерывность:
  \(\varphi _{k}  \xrightarrow{D  (\mathbb{R} )}  0  \;  \Rightarrow   \; \alpha  \varphi _{k}  \xrightarrow{D  (\mathbb{R} )}  0\).
  \(\supp \varphi _{k}  \subset   [A, B]  \;  \Rightarrow   \;  (f, \alpha  \varphi _{k})  \rightarrow   0\).

  \begin{example}

  \begin{enumerate}
  \def\labelenumii{\arabic{enumii}.}
  \item
    \(\alpha  \delta \),
    \((\alpha  \delta , \varphi )  =  (\delta , \alpha  \varphi )  = \alpha (0) \varphi (0)\),
    \(x \delta   (x)  =  0\).
  \item
    \(\left(x P\frac{1}{x}, \varphi \right)  =  \left(P\frac{1}{x}, x \varphi \right)  =  \int \limits_{\mathbb{R} } \varphi (x) d x\),
    \(x P\frac{1}{x}  =  1\).
  \end{enumerate}

  \end{example}
\item
  \textbf{Замена переменной}

  Для регулярных: \[
  c: \mathbb{R} \rightarrow \mathbb{R} \in \operatorname{Bij}, c \in  C^\infty , c'(x) \neq   0\\
  \int \limits_\mathbb{R} f(c(x))\varphi (x)dx =\int \limits_\mathbb{R} f(y)\frac{\varphi   \left(c^{-   1}  (y)\right)}{\left|c'  \left(c^{-   1}  (y)\right)\right|}dy
  \]

  \begin{definition}

  \(C  \colon \mathbb{R}   \rightarrow  \mathbb{R}   \text{  -  биекция,  } c  \in  C^{\infty }  (\mathbb{R} )  \text{,  }  (\forall  x  \ c'(x)  \neq   0)  \text{,  } f  \in  D'  (\mathbb{R} )  \colon  (f  \circ  c, \varphi )  =  \left(f,  \frac{\varphi   \left(c^{-   1}  (y)\right)}{\left|c'  \left(c^{-   1}  (y)\right)\right|}\right)\).

  \end{definition}

  Корректность
  \(\varphi   \in  D  (\mathbb{R} )  \;  \Rightarrow   \; \varphi   \circ  c^{-   1}  \in  D  (\mathbb{R} )  \;  \Rightarrow   \;  \frac{\varphi   \circ  c^{-   1}}{\left|c'  \circ  c^{-   1}\right|}  \in  D  (\mathbb{R} )\).
  Линейность --- очевидно. Непрерывность
  \(\varphi _{k}  \xrightarrow{D  (\mathbb{R} )}  0  \;  \Rightarrow   \;  \frac{\varphi _{k}  \circ  c^{-   1}}{\left|c'  \circ  c^{-   1}\right|}  \in  D  (\mathbb{R} )\).

  \begin{example}

  \(c(x)  = \gamma  x\), здесь \(\gamma \neq 0\),
  \(c^{-   1}  (x)  =  \frac{x}{\gamma }\).
  \((\delta (\gamma  x), \varphi )  =  \left(\delta ,  \frac{\varphi   \left(\frac{x}{\gamma }\right)}{|\gamma |}\right)  =  \frac{1}{|\gamma |} \varphi   (0)\),
  \(\delta   (\gamma  x)  =  \frac{1}{|\gamma |} \delta   (x)\).

  \end{example}

  \begin{warning}

  \(c  \circ  f\) - не определяется ``ни в каком смысле''.

  \end{warning}
\item
  \textbf{Дифференцирование}

  Регулярные: \[
  \int\limits_R f'\varphi \cdot dx=f\varphi \Big|_{-\infty }^{\infty }-\int\limits_R f\varphi '\cdot dx=-\int\limits_R f\varphi '\cdot dx
  \]

  \begin{definition}

  \(f  \in  D'  (\mathbb{R} )  \text{,  } \varphi   \in  D  (\mathbb{R} )  \colon  (f', \varphi )  =  -   (f, \varphi ')\).

  \end{definition}

  Очевидны:

  \begin{itemize}
  \tightlist
  \item
    Корректность.
  \item
    Линейность.
  \item
    Непрерывность.
  \end{itemize}

  \begin{example}

  \begin{enumerate}
  \def\labelenumii{\arabic{enumii}.}
  \item
    \(\theta   (x)  =  \begin{cases}1  & x  >  0\\  0  & x  <  0\end{cases}\),
    \((\theta , \varphi )  =  \int \limits_{0}^{\infty } \varphi   (x) d x\),
    \((\theta ', \varphi )  =  -   (\theta , \varphi ')  =  -   \int \limits_{0}^{\infty } \varphi '  (x) d x  = \varphi   (0)\),
    \(\theta '  = \delta \).
  \item
    \((\delta ', \varphi )  =  -  \varphi '  (0)\).
  \item
    \((f^{(n)}, \varphi )  =  (-   1)^{n}  (f, \varphi ^{(n)})\).

    \begin{warning}

    Обобщенные функции бесконечно дифференцируемы в смысле \(D'\).

    \end{warning}
  \item
    \(\alpha   \in  C^{\infty }  (\mathbb{R} )  \text{,  } f  \in  D'  (\mathbb{R} )  \colon  (\alpha  f)'  = \alpha ' f  + \alpha  f'\).
    Пусть \(\varphi   \in  D  (\mathbb{R} )\), тогда:
    \(((\alpha  f)', \varphi )  =  -   (\alpha  f, \varphi ')  =  -   (f, \alpha  \varphi ')  =  -   (f,  (\alpha  \varphi )')  +  (f, \alpha ' \varphi )  =  (f', \alpha  \varphi )  +  (\alpha ' f, \varphi )\).
  \end{enumerate}

  \end{example}

  \begin{exercise}

  \begin{enumerate}
  \def\labelenumii{\arabic{enumii}.}
  \item
    \((\ln  (x))'  = P\frac{1}{x}\).
  \item
    \(f  \in  D'  (\mathbb{R} )  \text{,  } f'  =  0  \;  \Rightarrow   \; f  = const\).
  \item
    \(f  \in  D'  (\mathbb{R} )  \text{,  } f^{(m)}  =  0  \;  \Rightarrow   \; f  \text{  -  многочлен  }  \deg f  < m\).
  \item
    \(f\) с разрывом в \(0\),
    \(\left.f\right|_{(-   \infty ,  0)}  \in  C^{1}  ((-   \infty ,  0])\),
    \(\left.f\right|_{(0,  \infty )}  \in  C^{1}  ([0,  \infty ))\)
    \(f(+  0)  -  f(-   0)  = h\),
    \(f_{D'}'  = f_{\text{классическая}}'  + h \delta   (x)\).
  \end{enumerate}

  \end{exercise}
\item
  \textbf{Сходимость}

  \begin{definition}

  \(f_{k}  \xrightarrow{D'  (\mathbb{R} )} f  \;  \Leftrightarrow   \;  (f_{k}, \varphi )  \rightarrow   (f, \varphi )  \;  \forall  \varphi   \in  D  (\mathbb{R} )\).

  \end{definition}

  \(f_{k}  \rightarrow  f\), \(g_{k}  \rightarrow  g\),
  \(\alpha , \beta   \in  \mathbb{C} \),
  \(\alpha  f_{k}'  + \beta  g_{k}'  \rightarrow  \alpha  f  + \beta  g\).
  \(f_{k}  \xrightarrow{D'} f  \;  \Rightarrow   \; f_{k}'  \xrightarrow{D'} f'\),
  \((f_{k}', \varphi )  =  -   (f_{k}, \varphi ')  \rightarrow   -   (f, \varphi )  =  (f', \varphi )\).

  \begin{exercise}

  \(\cos  (k x)  \xrightarrow{D'}  0\).

  \end{exercise}

  \begin{theorem}

  \(f_{k}  \in  L_{1}  (\mathbb{R} )\):

  \begin{enumerate}
  \def\labelenumii{\arabic{enumii}.}
  \item
    \(f_{k}  (x)  \geqslant  0\) почти всюду.
  \item
    \(\forall  a  >  0  \;  \int \limits_{-  a}^{a} f_{k}  (x) d x  \rightarrow   1\).
  \item
    \(\int \limits_{\mathbb{R} } f_{k} d x  =  1\).
  \end{enumerate}

  Тогда: \(f_{k}  \xrightarrow{D'} \delta   (x)\).

  \end{theorem}

  \begin{proof}

  \emph{Proof.} \(\varphi   \in  D  (\mathbb{R} )\)
  \(\varepsilon   >  0  \;  \exists  \alpha   >  0  \colon  |\varphi (x)  -  \varphi (0)|  <  \frac{\varepsilon }{2}  \text{ при  }  |x|  \leqslant \alpha \).
  \(\left|\int \limits_{\mathbb{R} } f_{k}  (x) \varphi (x) d x  -  \varphi (0)\right|  =  \left|\int \limits_{\mathbb{R} }  (f_{k}  (x) \varphi   (x)  -  \varphi (0)) d x\right|  \leqslant  \left|\int \limits_{|x|  < \alpha }  (f_{k}  (x) \varphi   (x)  -  \varphi (0)) d x\right|  +  \left|\int \limits_{|x|  > \alpha }  (f_{k}  (x) \varphi   (x)  -  \varphi (0)) d x\right|  \leqslant  \frac{\varepsilon }{2}  \int \limits_{-  \alpha }^{\alpha } f_{k}  (x) d x  +  2  \max\limits_{\mathbb{R} }  |\varphi |  \int \limits_{|x|  > \alpha } f_{k}  (x) d x  \rightarrow   \frac{\varepsilon }{2}  \;  \Rightarrow   \;  \exists  K  \colon  \left|\int \limits_{\mathbb{R} } f_{k} \varphi _{k} d x  -  \varphi   (0)\right|  < \varepsilon   \text{ при  } k  > K  \;  \Rightarrow   \;  (f_k, \varphi )  \rightarrow  \varphi (0)  \;  \Rightarrow   \; f_{k}  \xrightarrow{D'} \delta \).

  \end{proof}

  \begin{warning}

  Верно
  \(\int \limits_{\mathbb{R} } f_{k} \varphi  d x  \rightarrow  \varphi (0)  \;  \forall  \varphi   \text{  -  непрерывная и ограниченная на  } \mathbb{R} \).

  \end{warning}

  \begin{exercise}

  \(f  \in  L_{1,  \mathrm{loc}}  (\mathbb{R} )  \colon  \text{1) и  2)}  \;  \Rightarrow   \; f_{k}  \xrightarrow{D'} \delta \).
  (третий пункт можно убрать)

  \end{exercise}

  \begin{warning}

  Такие последовательности - это \(\delta \) - образные
  последовательности.

  \end{warning}

  \begin{example}

  \begin{enumerate}
  \def\labelenumii{\arabic{enumii}.}
  \item
    \(f_{k}  (x)  =  \begin{cases}k  & x  \in   \left(0,  \frac{1}{k}\right]\\  0  &  \text{иначе}\end{cases}\),
    \(f_{k}  \rightarrow  \delta \).
  \item
    \(f_{\varepsilon }  (x)  =  \frac{\varepsilon }{\pi   (x^2  + \varepsilon ^2)}\).
    \(\int \limits_{-  a}^{a}  \frac{\varepsilon }{\pi   (x^2  + \varepsilon ^2)} d x  =  \left.\frac{1}{\pi }  \arctg  \left(\frac{x}{\varepsilon }\right)\right|_{-  a}^{a}  =  \frac{2}{\pi }  \arctg  \left(\frac{a}{\varepsilon }\right)  \rightarrow   1\).
    \(\int \limits_{\mathbb{R}  f_{\varepsilon }} d x  =  1  \;  \Rightarrow   \; f_{\varepsilon }  \rightarrow  \delta \).
  \item
    \(f_{t}  (x)  =  \frac{e^{-   \frac{x^2}{t}}}{\sqrt {\pi  t}}  \text{, где  } t  >  0\).
    \(\frac{1}{\sqrt {\pi  t}}  \int \limits_{-  a}^{a} e^{-   \frac{x^2}{t}} d x  =  \frac{1}{\sqrt {\pi }}  \int \limits_{-   \frac{a}{\sqrt {pi}}}^{\frac{a}{\sqrt {\pi }}} e^{-  y^2} d y  \rightarrow   1\).
    \(\frac{1}{\sqrt {\pi  t}}  \int \limits_{-  a}^{a} e^{-   \frac{x^2}{t}} d x  =  1\).
  \end{enumerate}

  \end{example}

  \begin{exercise}

  \(\frac{\sin N x}{\pi  x}  \rightarrow  \delta   (x)\).

  \end{exercise}
\item
  Носитель \textbf{обобщённой} функции

  Локальный класс \(D(a, b)  = C_0^{+  \infty }  (a, b)\) (у нас
  определение опускалось, определяли сразу на \(\mathbb{R} ^n\))

  \begin{definition}

  \(\varphi _k  \overset{D(a, b)}{\Longrightarrow} \varphi   \Leftrightarrow \)

  \begin{enumerate}
  \def\labelenumii{\arabic{enumii}.}
  \item
    \(\exists  c, d: a  < c  < d  < b:  \  \supp \varphi _k  \subset   [c, d]\)
  \item
    \(\varphi _k^{(l)}  \to \varphi ^{(l)}\) равномерно на \([a, b]\)
  \end{enumerate}

  \end{definition}

  \(D'(a,b)\) -- пространство непрерывных функционалов над пространством
  функций \(D  (a,b)\)

  \begin{definition}

  \(f  \subset  D'(\mathbb{R} )  \ f|_{(a,b)}  =  0  \Leftrightarrow   (f, \varphi )  =  0  \  \forall  \varphi   \in  C_0^{+\infty }  (a, b)\)

  Или эквивалентное (без использования локального класса)
  \(\Leftrightarrow  (f, \varphi )=0\ \forall \varphi \in D(\mathbb{R} ): \supp {\varphi }\in (a, b)\)

  \end{definition}

  \begin{definition}

  \(\operatorname{supp} f  = \mathbb{R}   \setminus  \{ x: f  =  0  \text{ в окрестности  } x  \}\)

  \end{definition}

  \(\Rightarrow \supp f\) -- замкнутое множество

  \begin{lemma}

  \(\exists  f  \in  D'(\mathbb{R} ),  \  \operatorname{supp} f  \subset   [A, B]\)
  \(\exists  \psi   \in  D(\mathbb{R} ),  \  \operatorname{supp} f  \subset   [C, D]\)
  \([A, B]  \cap   [C, D]  =  \varnothing  \Rightarrow   (f, \psi )  =  0\)

  \begin{proof}

  \emph{Proof.} Пусть \(x\in [C;D]\Rightarrow \exists \delta >0: f=0\) в
  \((x-\delta , x+\delta )\Rightarrow \exists N: [C;D]\subset ⋃\limits_{j=1}^{N}(x_j-\delta _j, x_j+\delta _j)\).

  \(\exists \) разбиение единицы:
  \(\{\eta _j\}^N_{j=1}: \eta _j\in C^\infty (\mathbb{R} )\), такое, что
  \(\supp \eta _j \subset  (x_j - \delta _j, x_j + \delta _j)\),
  \(\sum_{j=1}^{N}\eta _j=1\ \forall x\in [C;D]\). Тогда \[
  \psi (x) = \sum\limits_{j=1}^N\eta _j\psi ;\quad (f, \psi ) = \sum_{j=1}^N(f, n_j\psi )=0\\
  n_j\in D(\mathbb{R} );\;\supp (n_j\psi )\subset (x_j-\delta _j, x_j + \delta _j) 
  \]

  \end{proof}

  \end{lemma}

  \begin{theorem}

  \(\exists  f  \in  D'(\mathbb{R} ),  \  \operatorname{supp} f\) --
  компакт
  \(\Rightarrow   \exists  m  \in  \mathbb{N} _0,  \  \exists  g  \in  C(\mathbb{R} ): f  = g^{(m)}\)
  (в смысле \(D'(\mathbb{R} )\))

  \begin{proof}

  \emph{Proof.} Без доказательства.

  \end{proof}

  \end{theorem}

  \begin{exercise}

  \(\exists  f  \in  D'(\mathbb{R} ):  \ f  \neq  g^{m}  \  \forall  m  \in  \mathbb{N} _0,  \  \forall  g  \in  C(\mathbb{R} )\)

  \end{exercise}

  \begin{theorem}

  \(\exists  f  \in  D'(\mathbb{R} ),  \  \operatorname{supp} f  =  \{0\}  \Rightarrow   \exists  m  \in  \mathbb{N} _0,  \  \exists  c_0, c_1,  \dots , c_m  \in  \mathbb{C} : f  =  \sum \limits_{j  =0}^m c_j \delta ^{(j)}\)

  \begin{exercise}

  Доказать теорему \textbf{3}, используя теорему \textbf{2} и упражнения
  \textbf{3} и \textbf{4} из секции 3 (дифференцирование)

  \end{exercise}

  \end{theorem}
\end{enumerate}
